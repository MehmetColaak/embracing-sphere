\chapter{Conclusion}
    \section{Summary of Findings}
        The experience of researching and experimenting was no less than great for me. Throughout the project realization, I learned and tested many different techniques. This long marathon of writing about my research and Embracing Sphere while simultaneously assembling Embracing Sphere for the Ars Electronica Festival expanded my skillset.\par

        Embracing Sphere, an Environmental Storytelling with Audio-Tactile Playback System, started with the question of "how can environmental storytelling be conveyed through non-visual mediums?". Later, it became more focused research about audio-tactile interfaces and their usage in creative works such as narration and environmental storytelling with the pursuit of a multi-modal sensory medium. My background in technical audio design helped me a lot when structuring this multi-modal playback system and deciding the technologies that I want to use in the Embracing Sphere.\par

        The considerations about the technical requirements of the work eventually shaped the results. Investigating the potential of audio-tactile systems for environmental storytelling, this thesis created an interdisciplinary grounding, mentioning game audio, sound design, human-computer interaction and artistic context this project is situated within.\par

        The theoretical foundation and examples in the Chapter 2, written to establish and support the argument that environmental storytelling can be efficiently communicated through multi-modal, especially audio-tactile, channels. For the novelty of the research, both of the modalities investigated with specific techniques, such as procedural room impulse response generation in the auditory domain and reverse engineered sampling of haptic signals from racing simulation games and solid surface vibrotactile recordings from games and real life structures.\par

        I hope the use of all the technologies and techniques that I investigated was satisfying when translated into my artistic output, Embracing Sphere. Most of the technical execution (the hardware and software integration) involved trials and errors, resulting in self-observatory decisions except for the headphone usage as an auditory display. Although artistic content (sound design, virtual environment, narrative structure) decisions are affected by the theme and the concept of the Ars Electronica Festival. One of my aims was to reach a format or theme agnostic system on both the hardware and software side of the project. Personally, I believe the current version of Embracing Sphere can be adapted and developed according to different needs and themes in the future.\par
    \section{Observations from Ars Electronica Festival}
        This section will cover Embracing Sphere installation exhibited at the Ars Electronica Festival and will show a collection of qualitative feedback through informal interviews and conversations with visitors of the Ars Electronica Festival. The opportunity to present the work to a public audience, earned through the Interface Cultures master's programme and the installation exhibited at the Kunstuni-Campus exhibition in the Post City building in Linz.\par

        To gather systematic feedback during the exhibition, I employed a combination of on-site observation, informal visitor interviews and conversations. Key audience reactions and behavioral patterns observed throughout the festival’s duration, highlighting initial engagement and notable user responses to the audio-tactile system.\par

        One very distinct observation that I have is that the average audience athleticism is worse than I expected. As the installation has a racing seat that requires sitting on to consume the experience, the metal frame and seat are placed close to ground level, about 35 cm high at sitting level. Many people struggled to stand up while seated that low without any bars to hold. This is my subjective interpretation but I also think many people didn't bother to try the experience for the same reason.\par

        An observation that I had expected was the spatial definition of the binaural mix in the audio playback. It was a downside, whilst I could have used a multichannel loudspeaker system for a more precise spatial feeling but as I stated in the previous chapters, the Ars Electronica Festival exhibition was a collaborative exhibition, so I had neighbours that I had to be careful not to be distracting.\par

        ...\par
    \section{Answering Research Questions}
        \begin{itemize}
            \item To what extent does multi-modality play a role in the environmental narration?
            \item How can multi-modal stimulation be utilized to enhance auditory environmental narration?
            \item How effectively can an interactive audio-tactile system convey distinct environmental characteristics (space size, material properties, etc.) and narrative cues to an audience?
        \end{itemize}

        This section will cover the three central research questions of this thesis, as indicated above. Beginning with an evaluation of how multi-modality informs environmental storytelling in the context of audio-tactile integration.\par

        The influence of multi-modal stimulation on auditory environmental storytelling is examined by reflecting on the system’s design and the observed effects of combining haptic and auditory feedback on immersion and spatial perception. According to personal observations, Embracing Sphere as a playback system was quite efficient in describing a virtual environment with audio-tactile channels. Although the environmental storytelling side was either hard to examine, since each audience is unique and their focus or critical listening capabilities may affect their interpretation of the story.\par

        The second research question is answered throughout the thesis in detail. I looked for the concepts that I can derive from real world immersive experiences, such as vibrating enclosures, surfaces and surroundings that we feel regularly in real-life (vehicle, bridge, underwater).\par

        Embedding story cues into these surroundings was the main challenge in that subject and from my point of view, I managed to embed audio-tactile information into the virtual environment. That information could have been missed or neglected by the audience but environmental storytelling, in essence, lives in this neglected domain that demands intentional exploration.\par

        In the 3rd research question, I can self direct the biggest criticism, which is that before this research, I was assuming that the space perception in the auditory domain is really intuitively connected with the reverb and reflection character of the space. One aspect I missed was that the time difference between the initial sonic event and the first reflected sound also gives highly distinctive cues about the size of the space. This concept is already known as "early reflections" or "pre-delay" in digital audio processing vocabulary. Further developments of this project will consider adding this feature to the procedural room impulse response generation. The rest of the features developed in the auditory channel worked well and managed to convey narrative cues to the audience, since it was inheriting most of the methods from cinema and game studies.\par

        According to the noticed issues in the system, further developments will be focused on spatial precision and ergonomics.\par

        Ultimately, a multi-modal system developed under the research of environmental storytelling, audio and haptics. With the support of many exterior sources, I pursued the potential of this combination. From an academic standpoint, potential examination still requires more qualitative and quantitative evaluation methods, while from an artistic researcher standpoint, I hope to establish a new interface, transmitting the information to humans in a novel way. Embracing Sphere as an interface for a newborn and it is going to be developed further with many new content and technological improvements.\par