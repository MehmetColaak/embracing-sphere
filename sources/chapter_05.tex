\chapter{Evaluation}
    \section{Evaluation Goals} What aspects of the system are you evaluating (e.g., technical performance, effectiveness of conveying space/material, narrative impact, user engagement)?
    \section{Methodology} Describe your evaluation approach. Examples:
        \subsection{Perceptual Effectiveness} Can participants understand environmental changes or narrative shifts based on haptic/audio-only cues?
        \subsection{Narrative Comprehension} Participants reconstructing narrative from stimulus to see how well the system conveys narrative arcs without visuals.
        \subsection{Qualitative Analysis} Interviews/focus groups analyzing richness, imaginative engagement, or associations evoked.
        \subsection{Technical Performance} Latency between tactile/audio triggers, fidelity of convolution reverb simulation, procedural realism of RIRs.
    \section{Results} Present the collected data (quantitative and/or qualitative).
    \section{Discussion} Interpret the results in relation to your research questions and goals. What worked well? What were the limitations? How did users respond to the audio-tactile combination?