\chapter{Conclusion}
    \section{Summary of Findings}
        The experience of researching and experimenting was no less than a great for me. Throughout the project realization I learned and tested many different techniques. This long marathon of writing about my research and Embracing Sphere while simultaneously assembling Embracing Sphere for Ars Electronica Festival, expanded my skillset.\par

        Embracing Sphere, an Environmental Storytelling with Audio-Tactile Playback System, started with the question of "how an environmental storytelling can conveyed through non-visual mediums?". Later it became more focused research about audio-tactile interfaces and their usage in creative works such as narration and environmental storytelling with the pursue of multi-modal sensory medium. My background in technical audio design helped me a lot when structuring this multi-modal playback system and deciding the technologies that I want to use in the Embracing Sphere.\par

        The considerations about technical requirements of the work, eventually shaped the results. Investigating the potential of audio-tactile systems for environmental storytelling, created an interdisciplinary grounding in the thesis, mentioning about game audio, sound design, human-computer interaction and artistic context this project situated within.\par

        The theoretical foundation and examples in the Chapter 2, written to establish and support the argument that environmental storytelling can be efficiently communicated through multi-modal, especially audio-tactile, channels. For the novelty of the research both of the modalities investigated with specific techniques such as procedural room impulse response generation in auditory domain and reverse engineered sampling of haptic signals from racing simulation games and solid surface vibrotactile recordings from games and real life structures.\par

        I am satisfied that I would be able to use all the technologies and techniques that I investigated get translated in my artistic output, Embracing Sphere. Most of the technical execution (the hardware and software integration) situated with trials and errors resulted with self-observatory decisions except the headphone usage as auditory display. Although artistic content (sound design, virtual environment, narrative structure) decisions affected by the theme and the concept of the Ars Electronica Festival. One of my aims was to reach format or theme agnostic system in both hardware and software side of the project. Personally I believe, current version of Embracing Sphere can be adapted and developed according to different needs and themes in the future.\par
    \section{Observations from Ars Electronica Festival}
        This section will cover Embracing Sphere installation exhibited at the Ars Electronica Festival and will show a collection of qualitative feedback through informal interviews and conversations with visitors of Ars Electronica Festival. The opportunity to present the work to a public audience, earned via Interface Cultures master's programme and the installation exhibited at Kunstuni-Campus exhibition in the Post City building in Linz.\par

        To gather systematic feedback during the exhibition, I employed a combination of on-site observation, informal visitor interviews and conversations...\par

        Key audience reactions and behavioral patterns observed throughout the festival’s duration, highlighting initial engagement and notable user responses to the audio-tactile system...\par

        Insights noticed about the effectiveness, limitations and potential of Embracing Sphere as experienced in a live, public setting.\par
    \section{Answering Research Questions}
        \begin{itemize}
            \item To what extent does multi-modality play a role in the environmental narration?
            \item How multi-modal stimulation can be utilized to enhance auditory environmental narration?
            \item How effectively can an interactive audio-tactile system convey distinct environmental characteristics (space size, material properties etc.) and narrative cues to an audience?
        \end{itemize}

        This section will cover the three central research questions of this thesis indicated above. Beginning with an evaluation of how multi-modality informs environmental storytelling in the context of audio-tactile integration.\par

        The influence of multi-modal stimulation on auditory environmental storytelling is examined by reflecting on the system’s design and the observed effects of combining haptic and auditory feedback on immersion and spatial perception. According to personal observations, ...\par

        An assessment of the interactive audio-tactile system’s ability to convey environmental and narrative cues is provided, with a focus on practical achievements, identified limitations, and areas where further improvement is necessary. ...\par