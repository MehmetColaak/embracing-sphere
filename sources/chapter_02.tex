\chapter{Background and Related Works in the Field}
    \section{Environmental Narration}
        \subsection{Definition of Environmental Narration} Define "environmental narration" (storytelling through space, atmosphere, implicit cues).
        \subsection{Existing Methods in Different Mediums} Discuss existing techniques in games, film, sound art, and literature (e.g., sound design, level design, environmental puzzles, use of acoustics).
        \subsection{Auditory Environmental Narration} Give an example from the Game Return of the Obra Dinn. 
    \section{Room Acoustics}
        \subsection{Definitions for Room Acoustics} Explain Room Impulse Responses (RIRs): what they are, what information they contain (reverberation, spatial cues).
        \subsection{Room Impulse Response Measurement Methods} Overview of RIR measurement techniques.
        \subsection{Convolution in Math and Digital Audio} Explain Convolution operation and usage in audio-visual content.
        \subsection{Room Acoustics in Sound Art} Explain Alvin Lucier - "I Am Sitting in a Room" Process-based art, using the room's acoustics as both medium and subject, the iterative feedback loop revealing resonant frequencies.
    \section{Procedural Audio Generation}
        \subsection{Procedural Content in Video Games} Define procedural content generation and give examples from games and interactive applications.
        \subsection{Procedural Audio and Synthesis} Explain methods of generating audio procedurally.
        \subsection{Integrating Procedural Generation into RIR} Introduce a method for generating RIR's procedurally.
    \section{Haptics and Perception}
        \subsection{Overview of Haptics} Explain different haptic feedback modalities (vibrotactile, force feedback, etc.) relevant to your project.
        \subsection{Human Tactile Perception} Briefly cover human tactile perception (how we sense texture, vibration, impact).
        \subsection{Usage} Examples of haptics in HCI, VR/AR, accessibility, and gaming.
    \section{Audio-Tactile Interaction}
        \subsection{Definition of Audio-Tactile Multi-Modal Interaction} Define audio-tactility and explain differences from other multi-modal systems.
        \subsection{Usage of Audio-Tactility in Media} Examples of systems or experiences integrating audio and haptics.