\chapter{Introduction}
    \section{Research Question} How multi-modal stimulation can be utilized to enhance auditory environmental narration?
    \section{Multi-Modality} How does the integration of audio and tactile feedback influence the perception and interpretation of environmental narratives in an interactive art context?
    \section{Environment as a Tool for Narration} State clearly what your project sets out to achieve (e.g., "To design, implement, and evaluate an interactive audio-tactile system capable of procedurally generating environmental snapshots for narrative purposes").
    \section{Embracing Sphere and Perception} Briefly outline the novelty of your work (e.g., the specific procedural generation method, the integration approach, the application to narrative via an art installation).
    \section{Scope and Limitations} Define the boundaries of your research (e.g., types of environments modeled, specific haptic modalities used, complexity of the narrative, scale of the user evaluation if any).
    \section{Thesis Outline} Briefly state what each subsequent chapter will cover.