\begin{abstract}
    Conveying a narrative through environment is a known technique in multiple media domains\cite{Slow_Narrative}. Heavy reliance on visual cues, static soundscapes and easy grasp interfaces defines the limitations of current methods in environmental storytelling. Utilizing multi-modal stimuli and 3D audio volumes for enhancing environmental narration remains niche and open for new perspectives in storytelling and narration through the environment. 

    To explore possibilities beyond these limitations, this research explores the integration of haptic feedback and advanced acoustic modeling methods. Haptic feedback systems mostly used by video games, racing simulations and interactive art installations, are often fed by low frequency audio signals to create pulses on haptic actuators. Complementing tactile channel, Room Impulse Response (RIR) is a method is a method of capturing acoustic properties of an enclosed volume/space later to use in reverberation(specifically convolution reverbs) to reconstruct the same acoustic responses while simulating different materials. Combining these distinct modalities like tactile pulses and acoustic reconstruction offers novel ways to represent environmental snapshots purely through non-visual interfaces.
    
    Building on this potential, a multi-modal stimulation scenario using simultaneous audio signal playback driven by a speaker array incorporating procedurally generated RIRs and a bass shaker (transducer) will be this thesis's foundational method for creating artificial environments and allowing the audience to perceive the world.
    
    This applied work master’s thesis will investigate the current state and the potential of such multi-modal experiences in conveying environmental information and narrative, aiming to design, implement and evaluate an interactive audio-tactile system capable of procedurally generating environmental snapshots for narrative purposes.
\end{abstract}