\chapter{Personal Project, System Design and Methodology}
    \section{Conceptual Framework}
        This section details the concept and design features of the Embracing Sphere installation. It describes the main conceptual components chosen and explains how they integrate to create an audio-tactile environmental storytelling experience.\par

        Embracing Sphere aims to combine key theoretical constructs that have been covered in Chapter 2 (environmental storytelling, multi-modal perception, and audio-tactile interaction). This vision directs the design and the implementation of the Embracing Sphere.\par

        \begin{figure}[H]
            \centering
            \includegraphics[width=0.8\textwidth]{images/venn_diagram.png}
            \caption{Venn diagram of conceptual elements of the Embracing Sphere.}
            \label{fig:VENN}
        \end{figure}        

        The above diagram shows conceptual elements of the Embracing Sphere with "technical" and "artistic" categorizations. This diagram shows my consideration points when creating the experience such as "How audio-tactile system is going to displayed to the audience?" and "How much narrative should I enforced in the experience?". These considerations basically constrain my technical and artistic decisions according to the requirements in exhibition thematically and technically.\par

        For example the audio-tactile display equipments is chosen with consideration of collaborative exhibition setting of Ars Electronica Festival Campus Exhibition. Therefore instead of a speaker array, a headphone and 3 bass shakers are chosen for audio-tactile play system. In this setup headphone is displaying binaural mix of audio channels composed in ambisonic format for 3D audio.\par
    \section{Procedural RIR Generation System}
        This section will detail the procedural RIR generation process in Embracing Sphere. Every different surrounding of us has different acoustical parameters and in time domain has different characters. Some enclosures has more dull and short response while other can be bright and long. This characteristic differences has covered in section 2.2.2 Room Impulse Response Measurement Methods.\par

        Utilizing this characteristic differences may apply contrast and environmental shifting abilities to auditory face of Embracing Sphere. Room impulse responses can be archived and used in demand with a RIR bank system that changing RIR file in convolution reverb but to artistically I chose more aleatoric way of generating RIR files with some parameters introduced in advance (procedural generation).\par

        Sabine formula that covered in 2.2.1 Definitions for Room Acoustics is simple enough to implement and outputs big enough range values to hear the difference. The RT60 parameter generated with sabine formula is used in RIR generation module.\par
        
        $$T_{60} = \frac{0.161 \cdot V}{A}$$
        
        In detail, sabine formula needs effective surface area (A), volume (V) variables to output RT60 value. Randomly generating numbers for an imaginary room dimensions can enable us to calculate an effective surface area and volume.\par

        \begin{figure}[H]
            \centering
            \includegraphics[width=0.8\textwidth]{images/max_RT60.png}
            \caption{RT60 value generator with randomized room dimentions and absorption parameter. Developed in Max/MSP visual scripting software.}
            \label{fig:RT60_MAX}
        \end{figure}
        
        As seen in the figure \ref{fig:RT60_MAX} 4 random number (3 for dimensions of the room, 1 for overall absorption coefficient) generated using Max/MSP random number generator object, with the scale object parameters the output numbers limited between 3 meter to 20 meter in width and depth and 3 meter to 4.5 meter for height of the imaginary room. Then rest of the expression objects are implementing Sabine formula into module. The output defines the length of the RIR file.\par

        RT60 alone is not enough to create a RIR. At the same time we need to generate a gradual fade and filter out envelopes to imitate an absorption slope. Subtractive synthesis came into formula at this point, basically gradually filtering and turning down the volume of a white noise with subtractive synthesis approach can be utilized in procedural RIR generation.\par

        Flowchart shown in the figure \ref{fig:FLOW} showing high level process flow in the procedural RIR generation. The output of this process is not a simulation level or mentioned RIR capturing methods level realistic but this primitive approach generates fast realtime RIR files to use in convolution reverb.\par
        
        \begin{figure}[H]
            \centering
            \includegraphics[width=0.8\textwidth]{images/procedural_RIR_flowchart.png}
            \caption{Flowchart of procedural RIR generation module.}
            \label{fig:FLOW}
        \end{figure}

        With high level explanation of procedural RIR generation, next section will cover haptic and audio content generation and playback system details.\par
    \section{Audio-Tactile Content Design and Playback System} How are the RIRs applied (e.g., real-time convolution)? What audio engine/libraries are used? Output format (stereo, binaural, ambisonics)?
    \section{Narrative Structure} Explain how the environmental snapshots are used to convey the intended narrative.